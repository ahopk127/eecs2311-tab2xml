% Created 2021-02-28 Sun 22:57
% Intended LaTeX compiler: pdflatex
\documentclass[11pt]{article}
\usepackage[utf8]{inputenc}
\usepackage[T1]{fontenc}
\usepackage{graphicx}
\usepackage{grffile}
\usepackage{longtable}
\usepackage{wrapfig}
\usepackage{rotating}
\usepackage[normalem]{ulem}
\usepackage{amsmath}
\usepackage{textcomp}
\usepackage{amssymb}
\usepackage{capt-of}
\usepackage{hyperref}
\date{\today}
\title{}
\hypersetup{
 pdfauthor={},
 pdftitle={},
 pdfkeywords={},
 pdfsubject={},
 pdfcreator={Emacs 27.1 (Org mode 9.4.4)}, 
 pdflang={English}}
\begin{document}

\tableofcontents

\newpage\\
\section{Front-End Tests}
\label{sec:orga8627b2}
Many of the front-end tests are implemented using a \texttt{ViewBot}, a class that simulates a GUI.\\
\subsection{PresenterTest}
\label{sec:org28a0400}
These test cases were derived by making one test for each important method of \texttt{Presenter}.\\
\begin{itemize}
\item \texttt{testConvert()}: Tests that the Presenter's convert method correctly interacts with the input (gets input text \& instrument, uses the parser to convert the tab and sets the output correctly).  It does not check that the converted MusicXML is correct - that is the backend tests' job.\\
It is implemented by first loading a sample input from a file and converting it using the parser to get the expected output.  Then, the ViewBot simulates inputting this text and selecting the correct instrument (guitar).  Then, the presenter's convert() method is called.  Finally, the ViewBot's output text is compared with the expected output.\\
\item \texttt{testConvertAndSave()}: Tests that the Presenter's \texttt{convertAndSave()} method works properly.  It is basically a combination of \texttt{testConvert} and \texttt{testSaveToFile}.  In addition, it tests that the boolean argument of \texttt{convertAndSave} works properly.\\
It is implemented by loading sample input and issuing a convertAndSave command, similarly to \texttt{testConvert}.  Then, both the View's output and the test file used as output have their contents checked with the correct text.  Like in \texttt{testConvert}, this output is generated by the backend code.\\
\item \texttt{testLoadFromFile()}: Tests that the "Load from File" command works properly.\\
It is implemented by using sample text in a file.  A presenter uses its \texttt{loadFromFile()} command to load the text into a ViewBot, and the ViewBot's input text is compared with the text that was in the file.\\
\item \texttt{testSaveToFile()}: Tests that the "Save to File" command works properly.\\
Some text is put in the output of a ViewBot.  Then, a Presenter uses its \texttt{saveToFile()} command to save the text to a file.  The file's text is compared with the text that was inputted in the ViewBot.\\
\end{itemize}

This testing is sufficient because every method of the Presenter is covered by a test (except the constructor, which is trivial and has only one line of code).  The Presenter's methods are simple enough that only one test is necessary for each.\\
\subsection{PromptingTextAreaTest}
\label{sec:org8bce673}
These test cases were derived by making one test case for each of the important functionalities of the \texttt{PromptingTextArea}: the colour and font of the prompt, the prompt text disappearing when the text box is focused, typing text in the box, and setting the area's font.\\
\begin{itemize}
\item \texttt{testPromptColourFont()}: Tests that the text box's colour and font is set properly.\\
This test works by creating a PromptingTextArea, and setting custom fonts with \texttt{setRegularFont()} and \texttt{setPromptFont()}.  The prompt is disabled, and the active colour and font is checked for correctness.  The prompt is enabled, and the same checks are performed.\\
\item \texttt{testPromptFocusChanges()}: Tests that the text box reacts properly to focus changes.\\
This test is implemented by simulating gaining and losing focus on the text box, and testing that the text box's text updates correctly.\\
\item \texttt{testPromptTyping()}: Tests that the text box reacts properly to typing and when methods are run on it (setText and setPromptText).\\
This test works by undergoing several operations (adding and deleting text, gaining and losing focus) while checking the text in the box is correct after each step.  Typing and deleting text is simulated using the setText method.\\
\item \texttt{testSetFont()}: Tests that fonts are set correctly.\\
This test is implemented by disabling the prompt, and setting the font.  The regular, prompt and active fonts are checked for correctness.  Then, the prompt is enabled and the regular, prompt and active fonts are checked again.  The test is repeated, but the prompt starts enabled and is switched to disabled in the second step.  This is done because the setFont method behaves differently based on whether the prompt is enabled or disabled.\\
\end{itemize}

This testing is sufficient because every method of the PromptingTextArea is tested at least once, and all important or complex methods have tests dedicated to them and their related methods: \texttt{testPromptColourFont} tests setRegularFont() and setPromptFont(); \texttt{testPromptTyping} tests setText() and setPromptText(); \texttt{testSetFont} tests setFont().  All other public methods are simple getters or setters, or methods that trivially call one of the tested methods.  In addition, the gain or loss of focus, an important feature of the \texttt{PromptingTextArea}, has its own dedicated testing method.\\
\subsection{ViewTest}
\label{sec:org470c809}
These tests were derived by making one test for each major method of the View interface.\\
All tests in this section are run once per View supported by the program, and once for the ViewBot.  This ensures that all of the Views support every possible feature.  Any test in this section that requires use of an unimplemented optional method is skipped.\\
All three of these tests are implemented by setting the paramater to some value, then comparing the value set to the value returned by the appropriate get method.\\
\begin{itemize}
\item \texttt{testInputText()}: Tests that all of the standard views can correctly get and set their input text.\\
\item \texttt{testOutputText()}: Tests that all of the standard views can correctly get and set their output text\\
\item \texttt{testInstrumentSelection()}: Tests that all of the standard views can correctly get and set their instrument selection\\
\end{itemize}

This testing is sufficient because, like in the Presenter, every important method in the View interface is tested by one test, except \texttt{showErrorMessage(String, String)}.  The showErrorMessage method cannot be tested automatically (because I do not want to specify \textbf{how} an error message is shown, only that one is shown), and it is trivial enough that I am not worried about it breaking (As of the time this document was written, all implementations of this method have only one line of code).  The View's methods are also simple enough that only one test per View is needed for each method.\\
\section{Back-End Tests}
\label{sec:org8056d4c}
\subsection{ParserTest}
\label{sec:org2b1c8aa}
These tests were derived by making sure that the parser was correctly interpreting the information provided through a text tab.\\
\begin{itemize}
\item \texttt{testScore()}: Tests that there is the correct amount of Staffs within a given Score.\\
This test was created by creating a string of a sample text tab, and using the Parse Tree to locate and count all of the staffs contained within the score, then comparing it with the expected amount of staffs.\\
\item \texttt{testStaff()}: Tests that there is the correct amount of Measures and number of strings in a given Staff.\\
\end{itemize}
This test was created by creating a string of a sample text tab of a single staff, and using the Parse Tree to locate and count every measure and every string, then comparing those with the number of expected measures and strings.\\
\begin{itemize}
\item \texttt{testTuning()}: Tests that each string in a staff is the expected tuning.\\
\end{itemize}
This test was created by creating a string of a sample text tab of a single staff, and using the Parse Tree to locate the tuning of each string and compare that with the expected tunings of each string. For inputs that do not contain string tunings, the expected tuning is the default guitar tuning.\\
\begin{itemize}
\item \texttt{stringItemCompareTo()}: Tests that the parser reads the notes in the correct order that they appear in the text tab.\\
\end{itemize}
This test was created by hard coding an array of different notes, with different positions in the tab, and adding them to an array, then comparing each note in the array to an array of each note in the order they are expected.\\

This testing is sufficient because there are tests for each basic component of a text tab (for example, measures or strings), and ensures that the parser is able to accurately interpret and store the information. Since the prototype is expected to handle simple tabs, only testing for the simplest components of a text tab were created. In the future this tester will have testing for more complex components, and testing for different components that the parser is not yet set up to interpret.\\
\subsection{NoteTest}
\label{sec:org03ceb62}
These tests were derived to make sure that note objects, which contain valuable information about notes that can be used in the xml conversion process, can be properly created.\\
\begin{itemize}
\item \texttt{noteTest()}: Tests that notes have the correct name and index.\\
\end{itemize}
This test was created by passing note to test, expected name of note and expected index of note as the parameter.\\
\begin{itemize}
\item \texttt{testToNote()}: There are 2 versions of testTonote, and both of them have different arguments. The first tests the toNote method in the Note class and checks if a valid note is correctly converted, and the other one tests the invalid notes.\\
\end{itemize}
There are 2 testTonote. The first was created by passing the string input(\emph{"tune + fret"}) and the string this note is on, and checks if this was a valid note and if it was converted correctly by comparing it to an expected note. The 2nd one was created just by passing string input(\emph{"tune + fret"}) and checks if an invalid note was entered by using exceptions.\\

This is sufficient testing because it checks that our system properly handles creating Note objects, which is a very important step in translating the information from text tabs to xml because notes are the main focus of learning songs through text tab. By testing the correctness and validity of these note objects, we can be sure that the notes that appear in a text tab will have the necessary information used in xml.\\
\end{document}
